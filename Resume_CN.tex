%-------------------------
% Resume in Latex
% Author : lokinko
% Based off of: https://github.com/jakeryang/resume
% License : MIT
%------------------------

% !TEX program = xelatex

\documentclass[letterpaper,11pt]{article}

\usepackage{latexsym}
\usepackage[empty]{fullpage}
\usepackage{titlesec}
\usepackage{marvosym}
\usepackage[usenames,dvipsnames]{color}
\usepackage{verbatim}
\usepackage{enumitem}
\usepackage[hidelinks]{hyperref}
\usepackage{fancyhdr}
\usepackage[english]{babel}
\usepackage{tabularx}
% only for pdflatex
% \input{glyphtounicode}
\usepackage{xeCJK}
\usepackage{ifplatform}  % 用于检测操作系统

% 跨平台中文字体设置:自动检测 Windows / macOS / Linux
\xeCJKsetup{AutoFakeBold=true, AutoFakeSlant=true}

        \setCJKmainfont{STSong}      % 对应英文字体中的 Serif
        \setCJKsansfont{STHeiti}    % 对应英文字体中的 Sans-serif (消除警告的关键)
        \setCJKmonofont{STHeiti}    % 对应等宽字体
        \xeCJKsetup{AutoFakeBold=true, AutoFakeSlant=true}
        \setCJKmainfont[AutoFakeBold=true, AutoFakeSlant=true]{STSong}
        \setCJKsansfont[AutoFakeBold=true, AutoFakeSlant=true]{STHeiti}
        \setCJKmonofont[AutoFakeBold=true, AutoFakeSlant=true]{STHeiti}

% \ifwindows
%     % Windows 字体
%     \setCJKmainfont{Microsoft YaHei}
%     \setCJKsansfont{Microsoft YaHei}
%     \setCJKmonofont{Microsoft YaHei}
% \else
%     \ifmacosx
%         % macOS 字体
%         \setCJKmainfont{STSong}      % 对应英文字体中的 Serif
%         \setCJKsansfont{STHeiti}    % 对应英文字体中的 Sans-serif (消除警告的关键)
%         \setCJKmonofont{STHeiti}    % 对应等宽字体
%         \xeCJKsetup{AutoFakeBold=true, AutoFakeSlant=true}
%         \setCJKmainfont[AutoFakeBold=true, AutoFakeSlant=true]{STSong}
%         \setCJKsansfont[AutoFakeBold=true, AutoFakeSlant=true]{STHeiti}
%         \setCJKmonofont[AutoFakeBold=true, AutoFakeSlant=true]{STHeiti}
%     \else
%         % Linux 字体 (需要安装 fonts-noto-cjk)
%         \setCJKmainfont{Noto Serif CJK SC}
%         \setCJKsansfont{Noto Sans CJK SC}
%         \setCJKmonofont{Noto Sans Mono CJK SC}
%     \fi
% \fi


% fontawesome
\usepackage{fontawesome5}

% fixed width
% \usepackage[scale=0.90,lf]{FiraMono}

% light-grey
\definecolor{light-grey}{gray}{0.83}
\definecolor{dark-grey}{gray}{0.3}
\definecolor{text-grey}{gray}{.08}

\DeclareRobustCommand{\ebseries}{\fontseries{eb}\selectfont}
\DeclareTextFontCommand{\texteb}{\ebseries}

% custom underilne
\usepackage{contour}
\usepackage[normalem]{ulem}
\renewcommand{\ULdepth}{1.8pt}
\contourlength{0.8pt}
\newcommand{\myuline}[1]{%
  \uline{\phantom{#1}}%
  \llap{\contour{white}{#1}}%
}


% custom font: helvetica-style
\usepackage{tgheros}
\renewcommand*\familydefault{\sfdefault} 
%% Only if the base font of the document is to be sans serif
\usepackage{fontspec}  % XeLaTeX 使用 fontspec 来处理字体


\pagestyle{fancy}
\fancyhf{} % clear all header and footer fields
\fancyfoot{}
\renewcommand{\headrulewidth}{0pt}
\renewcommand{\footrulewidth}{0pt}

% Adjust margins
\addtolength{\oddsidemargin}{-0.5in}
\addtolength{\evensidemargin}{0in}
\addtolength{\textwidth}{1in}
\addtolength{\topmargin}{-.5in}
\addtolength{\textheight}{1.0in}

\urlstyle{same}

\raggedbottom
\raggedright
\setlength{\tabcolsep}{0in}
\setlength{\footskip}{4.08003pt}

% Sections formatting - serif
% \titleformat{\section}{
%   \vspace{2pt} \scshape \raggedright\large % header section
% }{}{0em}{}[\color{black} \titlerule \vspace{-5pt}]

% TODO EBSERIES
% sans serif sections
\titleformat {\section}{
    \bfseries \vspace{2pt} \raggedright \large % header section
}{}{0em}{}[\color{light-grey} {\titlerule[2pt]} \vspace{-4pt}]

% only for pdflatex
% Ensure that generate pdf is machine readable/ATS parsable
% \pdfgentounicode=1

%-------------------------
% Custom commands
\newcommand{\resumeItem}[1]{
  \item\small{
    {#1 \vspace{-1pt}}
  }
}

\newcommand{\resumeSubheading}[4]{
  \vspace{-1pt}\item
    \begin{tabular*}{\textwidth}[t]{l@{\extracolsep{\fill}}r}
      \textbf{#1} & {\color{dark-grey}\small #2}\vspace{1pt}\\ % top row of resume entry
      \textit{#3} & {\color{dark-grey} \small #4}\\ % second row of resume entry
    \end{tabular*}\vspace{-4pt}
}

\newcommand{\resumeSubSubheading}[2]{
    \item
    \begin{tabular*}{\textwidth}{l@{\extracolsep{\fill}}r}
      \textit{\small#1} & \textit{\small #2} \\
    \end{tabular*}\vspace{-7pt}
}

\newcommand{\resumeProjectHeading}[2]{
    \item
    \begin{tabular*}{\textwidth}{l@{\extracolsep{\fill}}r}
      #1 & {\color{dark-grey}} \\
    \end{tabular*}\vspace{-4pt}
}

\newcommand{\resumeSubItem}[1]{\resumeItem{#1}\vspace{-4pt}}

\renewcommand\labelitemii{$\vcenter{\hbox{\tiny$\bullet$}}$}

% CHANGED default leftmargin  0.15 in
\newcommand{\resumeSubHeadingListStart}{\begin{itemize}[leftmargin=0in, label={}]}
\newcommand{\resumeSubHeadingListEnd}{\end{itemize}}
\newcommand{\resumeItemListStart}{\begin{itemize}}
\newcommand{\resumeItemListEnd}{\end{itemize}\vspace{0pt}}

\color{text-grey}

\hypersetup{
    colorlinks=true,      % 开启文字着色
    urlcolor=NavyBlue,        % 网页链接显示为蓝色
    linkcolor=NavyBlue,       % 内部引用显示为蓝色
    citecolor=NavyBlue        % 文献引用显示为蓝色
    pdfborderstyle={/S/U/W 1}   % 设置边框样式为下划线 (Underline),宽度为 1pt
}

%-------------------------------------------
%%%%%%  RESUME STARTS HERE  %%%%%%%%%%%%%%%%%%%%%%%%%%%%


\begin{document}

%----------HEADING----------
\begin{center}
  \textbf{\Huge 瞿祥谋} \\ \vspace{8pt}
  \small 
  \faPhone* \hspace{2pt} \texttt{(+86)15958040713} \hspace{8pt} $|$ \hspace{8pt}
  \faEnvelope \hspace{2pt} \href{mailto:lokinko.cs@gmail.com}{\texttt{lokinko.cs@gmail.com}} \hspace{8pt} $|$ \hspace{8pt}
  \faMapMarker* \hspace{2pt} \texttt{中国, 深圳} 
  \\ \vspace{4pt}
  \faGithub \hspace{2pt} \href{https://github.com/lokinko}{\texttt{lokinko()}} \hspace{8pt} $|$ \hspace{8pt}
  \faZhihu \hspace{2pt} \href{https://www.zhihu.com/people/lokinko}{\texttt{lokinko(1.4w followings)}} \hspace{8pt} $|$ \hspace{8pt}
  \faGraduationCap \hspace{2pt} \href{https://scholar.google.com/citations?user=kzBPdVgAAAAJ&hl=en}{\texttt{Google Scholar}}
  \\ \vspace{-3pt}
\end{center}

%
%-----------PROGRAMMING SKILLS-----------
\section{个人简介}
 \begin{itemize}[topsep=2pt, leftmargin=0in, label={}]
    \item \textbf{OPPO 研究院 | 高级算法工程师(2022.07 - 至今)}  
    \begin{itemize}[nosep, leftmargin=1.2em]
      \item 全链路参与GUI智能体项目整体搭建,主导智能体框架的技术探索与落地、构建端到端的GUI评估体系,作为技术负责人推进GUI智能体在自动打车等业务全量上线,参与数据飞轮方案优化与模型训练,构建通用、鲁棒的GUI智能体系统;
      \item 主导端云协同训练范式的系统设计与算法优化,作为项目主责推进海外端云推荐范式在广告推荐领域的落地并达成营收目标.
    \end{itemize}
    \item \textbf{重庆大学 | 硕士(2019.09 - 2022.06) \hspace{0.5cm}} \textbf{浙江工业大学 | 本科(2015.09 - 2019.06)}
  \end{itemize}


%-----------PROJECTS-----------

\section{项目经验}
  \resumeSubHeadingListStart
    \resumeSubheading
      {GUI Agent 自动执行}{2024.09 - 至今}{}{}
      \small 项目背景:构建系统级通用的自动执行能力,深入探索GUI Agent基模及框架能力,为用户解决「简单到不想做, 复杂到不会做」的日常需求,打造主动、个性化的 OS 级执行助手。\\ \vspace{3pt}

      \href{https://arxiv.org/abs/2507.16853}{[NeurIPS'25] MobileUse: A Hierarchical Reflection-Driven GUI Agent for Autonomous Mobile Operation}
      \begin{itemize}[nosep, leftmargin=1.2em]
          \item {[智能体框架] 主导构建基于 Planner-Executor-Hierarchical Reflector 的多智能体协同框架, 通过先验的主动探索能力构建 APP-specific 的通用经验, 并引入后延增强的 Hierarchical reflector 对不同粒度的GUI Agent轨迹进行反思,能迅速从多样的执行失败中恢复,提升智能体执行的鲁棒性。}
      \end{itemize}

      \href{https://arxiv.org/abs/2510.14621}{[WWW'26] ColorBench: Benchmarking Mobile Agents with Graph-Structured Framework for Complex Long-Horizon Tasks}
      \begin{itemize}[nosep, leftmargin=1.2em]
        \item{[GUI 自动化评测] 针对 GUI Agent 对长程任务执行能力的评估,设计一种基于 Graph-Structured Framework 的可扩展自动化评测方案,解决动态环境评估难复现、静态环境评估不充分,无法覆盖真实环境多路解的问题,以及真实测量多智能体错误恢复的问题。}
      \end{itemize}

      \href{https://arxiv.org/abs/2510.19386}{[Technical Report] ColorAgent: Building A Robust, Personalized, and Interactive OS Agent}
      \begin{itemize}[nosep, leftmargin=1.2em]
        \item{[OS Agent] 构建一种鲁棒、个性化、交互式的 OS Agent 框架,通过引入先验知识、后验增强、多模态交互等技术,实现对用户需求的精准理解、主动响应和持续优化。}
      \end{itemize}


        \resumeSubheading
        {端云协同 | 分布式机器学习}{2022.07 - 2024.09}{}{}
        项目背景:xxx
          \resumeItemListStart
            \resumeItem{[系统设计] 主导构建多智能体协同框架, 实现GUI Agent 自动执行}
            \resumeItem{[算法优化] 主导构建多智能体协同框架, 实现GUI Agent 自动执行}
            \resumeItem{[业务落地] 主导构建多智能体协同框架, 实现GUI Agent 自动执行}
          \resumeItemListEnd
          
  \resumeSubHeadingListEnd


%-----------AWARDS-----------
\section{获奖情况}
  \resumeSubHeadingListStart
    \resumeSubheading
      {北极星团队 | GUI 智能体}{2024.10}
      {OPPO研究院最高荣誉}{OPPO 研究院, 拓扑Lab}
      	\resumeItemListStart
    	\resumeItem {\textbf{Coursework}: Data Structures, Algorithms, Databases, Computer Systems, Machine Learning}
        \resumeItem 
            {\textbf{Research}: MIT Graybiel Lab (published author), MIT Media Lab (analyzed urban microbe spread)}
        \resumeItemListEnd

        \resumeSubheading
        {科技贡献奖 | 端云协同计算}{2025.12}
        {OPPO公司级技术荣誉}{OPPO 研究院, 同态Lab}
          \resumeItemListStart
        \resumeItem {\textbf{Coursework}: Data Structures, Algorithms, Databases, Computer Systems, Machine Learning}
          \resumeItem 
              {\textbf{Research}: MIT Graybiel Lab (published author), MIT Media Lab (analyzed urban microbe spread)}
          \resumeItemListEnd
  \resumeSubHeadingListEnd



%-------------------------------------------
\end{document}
